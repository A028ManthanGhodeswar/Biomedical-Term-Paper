\documentclass[12pt]{article}
\usepackage{setspace}
\usepackage{multicol}
\usepackage{hyperref}
\hypersetup{colorlinks = true, citecolor = blue, linkcolor = blue, urlcolor = blue}
\usepackage{mathptmx}

\usepackage{setspace}
\usepackage{graphicx}
\graphicspath{{images}}


\title{Project Report On : - 
\linebreak

Biomedical Tissue Engineering and Regenerative Medicine}

\begin{document}
\maketitle
\begin{figure}[h]
\centering
\includegraphics[scale=0.40]{Nitrr.png}
\end{figure}
\bigskip
\bigskip
\centering
\begin{large}
\title{National Institute of Technology, Raipur}
\end{large}
\medskip


\raggedright 
\begin{multicols}{2}
\author{Submitted By : Manthan Ajit Ghodeswar
\linebreak Roll no:- 21111028,
\linebreak Branch :- Biomedical Branch
\linebreak Semester :- First(2021-22)
\linebreak NIT Raipur, Chattisgarh}
\columnbreak
\columnbreak
\bigskip
\linebreak
Under the supervision of
\linebreak Dr.Saurabh Gupta
\linebreak Department of Biomedical Engineering,
\linebreak NIT Raipur, Chattisgarh
\end{multicols}

\clearpage

\centering
\tableofcontents
\clearpage

\section{Acknowledgment}
\setstretch{1.5}
\raggedright
\bigskip
\bigskip
I am grateful to Dr.Saurabh Gupta, Biomedical Engineering, for his proficient supervision on the term project "Biomedical Tissue Engineering". I am very thankful to you sir for your guidance and support.
\linebreak
\linebreak
\linebreak
\linebreak
\linebreak

\raggedleft
Manthan Ajit Ghodeswar
\linebreak
21111028
\linebreak
First Semester (2021-22), Biomedical Branch
\linebreak
National Institute of Technology, Raipur.
\linebreak
\linebreak
\linebreak

\raggedright
Date of Submission :  $7th  April, 2022.$
\clearpage

\centering
\section{Abstract}
\raggedright
\bigskip
\bigskip
In simple language, biomedical tissue engineering can be defined as the field of science involving growth of organs from implanting certain tissues and hence avoiding the transplantation processes. Regenerative Medicine can be briefed as a branch of medicine that constitutes of developing various methods to regrow, repair or replace damaged cells, tissues or even organs! But why is there a need of tissue engineering and regenerative medicine? The major problem with transplantation processes is that there's ambiguity on whether it will be successful or not. And if it is successful, we don't know how long it will last. This longevity issue arises due to our human body's immune system which thinks of the transplanted organ as a foreign object or an invader and thus tries to destroy it. In technical terms, this is called as immunological rejection. The two main factors in Tissue Engineering and Regenerative Medicine (TERM) are cell and scaffolds. A tissue scaffold basically provide structural support to facilitate cellular growth from the implanted tissue at a rapid pace.
\linebreak

\centering
\section{Introduction}
\raggedright
\bigskip
\bigskip
Eugene Bell is regarded as the father of tissue engineering. Tissue engineering basically started as a sub-field of biology but due to the advancements and rapid growth it was then declared as a new field itself. The reason for this growth is that TERM can combine various different fields of basic biosciences such as stem cell biology, functional scaffold materials, nanotechnology and even 3-D Printing (which also goes by additive manufacturing {AM}). TERM has the potential to play a significant role in increasing the longevity of a patient's life as it can even address issues as big as organ failure. There are mainly two types of cells used in TERM which namely are primary cells and stem cells. There are certain ways to fabricate tissue engineered scaffolds which are discussed in the given paper. The most basic requirement for a (3-D) cell or biomaterial complex is that it should support cell growth, facilitate proper transportation of nutrition and waste, and also aid in gas exchange.
\linebreak

\centering
\section{Discussion}
\raggedright
\bigskip
\bigskip
\subsection{Cells}
\bigskip
\bigskip
Cells being used in TERM should have strong ability of differentiation so as to ensure selective permeability. They should also have a stable passage and show appropriate response to stimuli and should be easily harvested and seeded into the extra - cellular matrix (ECM). Primary cells for TERM are extracted from the same organism which will undergo and accept the implantation. These cells do have relatively slower late of differentiation but are preferred because they undergo lower immunological response and also pathogen transmission. However, there are limited number of cells that are replaceable by tissue formation. For example, consider the spinal cord or the peripheral nerve etc. are simply too specific to be regenerated based on this method. This is because there are lot of difficulties encountered and complications arisen when we opt for direct biopsy. Also, there is a lack of replaceable primary cells. In such cases, stem cells are used.Stem cells can be explained as body's raw materials. They are very basic and fundamental group of cells from which further cells with specialized functions are generated. Theoretically speaking, implanting stem cells to an appropriate scaffold could form a related tissue. Stem cells, on the basis of stage, can be categorized into two parts :- Embryonic Stem Cells (ESCs) and Adult Stem Cells (ASCs).
\linebreak
\linebreak
\begin{tabular}{|c|c|c| }
\hline
Categories & Regenerated Tissues/ Repaired Tissues\\
Embryonic Stem Cells & Heart\\
& Kidney\\
&Bone/Cartilage\\
&Nerve\\
&Liver\\
&Lung\\
&Skin\\
&Ovary\\
&Retina\\
Adult Stem Cells &Skin\\
&Lung\\
&Adipose\\
&Bone/Cartilage\\
&Nerve\\
\hline
\end{tabular}

\subsection{Scaffolds}
\bigskip
\bigskip
Scaffolds provide a 3-D structure for cell differentiation into a tissue or an organ. The general requirements of scaffolds in TERM are that they should be enabling cell adhesion and proliferation. It's of no use if they can't stick together to function as a tissue. Further, to reduce immunogenic response, scaffolds must have good compatibility with the human body so as to reduce or almost nullify the damage it might take. Biodegradability of scaffolds is another thing to be taken into consideration. After the healing process, we need to make sure that the by-products after the biodegradation are non-cytotoxic.
\linebreak
\linebreak
\begin{figure}[h]
\centering
\includegraphics[scale=0.50]{Scaffold.png}
\end{figure}
\linebreak
Above are the images of various scaffolds of different shapes and sizes.
\clearpage

\centering
\section{Methodology}
\raggedright
Lots of various methods for preparation of scaffold have been introduced. Some of them have been discussed below :-

\subsection{Fiber Bonding}
Fiber bonding process of making scaffolds involves bonding unconnected fibers in the non-woven fabric to get a scaffold with stable structure. This can be done by two methods. The first method is the fibrous fixation technique and the second one is performed by spray coating on the surface of the fibers.

\subsection{Phase Separation / Freeze Drying}
Phase separation/freeze drying refers to giving rise to a steady-stable state of multi-component heterogeneous systems to the thermodynamically unstable which would tend to separate multi-phase systems thus reducing freeze energy under a specific condition. It can be done by Solvent Induced Phase Separation (SIPS) and Thermally Induced Phase Separation (TIPS).

\subsection{Solvent Casting / Particulate Leaching}
Solvent casting/particulate leaching is basically an improvised version of the fiber bonding method. In this method, the polymers are dissolved in organic solvents and then sodium chloride or polysaccharide-like crystals are added as the pore-forming agent. However, there is a high residue of organic solvents and also loss of loaded growth factors is seen while using this method.

\subsection{Gas Foaming}
As the name suggests, gas foaming uses gas as a porogen (particles that make pores) instead of using solid particles and avoids the organic solvent during the process. Since organic solvents and filtration processes are avoided in this process, the issue of residue is taken care of. However, the scaffolds formed by this process have a very high-proximity, i.e., they are very close structured meaning it has limited connections resulting in poor nutrient transportation, cell adhesion and migration.

\subsection{Rapid Prototyping Manufacturing}
Rapid Prototyping Manufacturing (RPM) uses CAD (Computer-Aided Design) model for rapidly manufacturing a complex 3-D physical entity. The biggest plus point of RPM is that it facilitates the creation of patient-specific customized scaffolds which is a big leap in personalized health care and precision medicine. However, customization is very costly, and the cost needs to go down to make it viable for clinical application.

\subsection{Electrospinning}
Electrospinning is a relatively modern and complex process of scaffold fabrication. The mechanism it is based on is that when liquid is charged under a condition of a certain high enough (threshold) voltage, there is some interaction between surface tension and electrostatic-repulsion that makes droplet at the tip of the spinneret (a cap or a plate having holes) erupt and stretch. In simple terms, electricity is used to create fibers from the solution itself.
\clearpage

\centering
\section{Conclusion}
\bigskip
\raggedright
Given below are the advantages and disadvantages of the various methods of scaffold fabrication : -
\linebreak
\linebreak
\begin{tabular}{|c|c|c|}
\hline
Methods & Advantages & Disadvantages\\
Fiber Bonding & Pores Connected Over Large & Stability is weak and high residue \\
&   Surface Area & of organic solvent\\
 & & \\
Phase Separation/ & Porosity can be controlled & Small Sized pores, weak, unstable\\
Freeze Drying & (No organic solvent) & structure and time consuming\\
& & \\
Solvent Casting/ & Simple to operate & Organic solvent residue\\
Particulate Leaching & Size of pores can be controlled & Loss of growth factors\\
& & \\
Gas Foaming & Controllable structure of scaffold & Weak mechanical strength\\
& Uniform, good connection of pores & Closed Pores\\
& & \\
Rapid Prototyping & Customized 3-D scaffolds & Special apparatus needed\\
Manufacturing (RPM) & (Precision Medicine) & Very Expensive\\
& & \\
Electrospinning & High Porosity & Weak Strength \\
 & Good homogeneity & Small - sized pores \\
 \hline
\end{tabular}
\clearpage

\centering
\section{Future Scope}
\raggedright
There is a lot of scope for TERM even in the future. The human brain cannot be regenerated exactly yet. Since the field focuses on studying cell and tissues on a very fundamental level very deeply, maybe someday even synthetic organs or tissues altogether can be manufactured. We could somehow figure out the regeneration of brain cells as we know that neurons do not undergo cell division. So if we can somehow figure that out, there will be a more than significant growth in our lifespan. If we can figure out the same for animals, we can stop the extinction and endangering the lives of other species as well. Finally, we can also use this study to figure out how do cancer cells divide and form the abnormalities that are encountered. If we are able to figure out how they reproduce, maybe by reverse-engineering the entire process, we could find a way to cure cancer itself.
\linebreak

\centering
\section{References}
\raggedright
\textbf{Zhi Li, Mao-Bin Xie, Yi Li, Yan Ma, Jia-Shen Li and Fang-Yin Dai.} Recent Progress in Tissue Engineering and Regenrative Medicine, Journal of Biomaterials and Tissue Engineering $6(10):755-766. (2016)$
\linebreak
\linebreak
$https://www.researchgate.net/figure/SEM-images-of-silk-scaffolds-A-C-and-a-CT-image-of-a-cynomolgus-monkey-L1-vertebra_fig9_6313407$

\end{document}